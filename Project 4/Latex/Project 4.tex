\documentclass[11pt]{article}

\usepackage{sectsty}
\usepackage{graphicx}
\usepackage{mathtools}

% Margins
\topmargin=-0.45in
\evensidemargin=0in
\oddsidemargin=0in
\textwidth=6.5in
\textheight=9.0in
\headsep=0.25in

\title{ Title}
\author{ Author }
\date{\today}

\begin{document}
%--Paper--

\section{Problem 1 a}

If \(r_i(t)\) and \(r_j(t)\) are correlated, then: 
$$r_i(t) = \alpha \: r_j(t)$$

Thus:
$$<r_i(t)r_j(t)>= \alpha \: <r_j(t)^2>$$
$$<r_i(t)><r_j(t)>= \alpha \: <r_j(t)>^2$$

In Addition:
$$(<r_i(t)^2> - <r_i(t)>^2) = (\alpha^2 <r_j(t)^2> - \alpha^2<r_j(t)>^2) = \alpha^2 (<r_j(t)^2> - <r_j(t)>^2)$$

Thus the denominator of \(\rho_{ij}\) is:
$$\sqrt[]{(<r_i(t)^2> - <r_i(t)>^2)(<r_j(t)^2> - <r_j(t)>^2)} = \sqrt[]{\alpha^2 (<r_j(t)^2> - <r_j(t)>^2)^2}$$

And the numerator of \(\rho_{ij}\) is:
$$<r_i(t)r_j(t)> - <r_i(t)><r_j(t)> = \alpha <r_j(t)^2> - \alpha <r_j(t)>^2 = \alpha (<r_j(t)^2> - <r_j(t)>^2)$$

And finally:

$$\rho_{ij} =\frac{\alpha (<r_j(t)^2> - <r_j(t)>^2)}{\sqrt[]{\alpha^2 (<r_j(t)^2> - <r_j(t)>^2)^2}} = \frac{\alpha}{\sqrt[]{\alpha^2}}$$

\begin{center}Meaning that if \(\alpha > 0\), \(\rho_{ij} = 1\) and if \(\alpha < 0\), \(\rho_{ij} = -1\) \end{center}

\section{Problem 1 b}

$$log[1+q_i(t)] =$$
$$log[\frac{p_1(t-1)}{p_1(t-1)} + \frac{(p_1(t)-p_1(t-1))}{p_1(t-1)}] =$$
$$log[\frac{p_1(t)}{p_1(t-1)}] =$$
$$log[p_1(t)] - log[p_1(t-1)]$$

\section{Problem 2 a}

$$w_{ij} = \sqrt[]{2(1-\rho_{ij})}$$

\newpage

\section{Problem 4}

$$\alpha = \frac{1}{|V|}\sum_{v_i\in V} P(v_i\in S_i)$$

The first method utilizes the minimum spanning tree (MST) generated in the previous question. \(Q_i\) is the set of neighbors from the MST that share the same sector, while \(N_i\) is the set of all neighbors in the MST:

$$P(v_i\in S_i) = \frac{|Q_i|}{|V_i|}$$


The second method effectively considers the entire graph. \(S_i\) is the total number of stocks in the same sector, while \(V\) is the set of all stocks (vertices):

$$P(v_i\in S_i) = \frac{|S_i|}{|V|}$$


Note that if we take the \textbf{first method of the full graph, these two values are equivalent.}
 Since the full graph is fully connnected:

$$Q_i = S_i,  N_i = V$$
Thus,

$$\alpha_1 = \alpha_2$$


\end{document}




